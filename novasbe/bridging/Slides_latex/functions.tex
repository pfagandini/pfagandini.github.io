\documentclass[aspectratio=169, handout]{beamer}
\usepackage[utf8]{inputenc}
\usepackage{amsmath}
\usepackage{tikz}
\usepackage{dsfont}
\usepackage{opensans}


\newtheorem{proposition}{Proposition}

\title[]{4509 - Bridging Mathematics}
\subtitle{Functions}
\author[P. Fagandini]{Paulo Fagandini}
\institute[]{}
\date{}

\usetheme{NOVASBE}

\begin{document}

\begin{frame}{Functions}
     A \emph{function} takes one element from a set, and associates it with an element of another set.
     
     \begin{definition}
        $f$ is a \textbf{function} from $A$ to $B$, if it links each element from $A$ to a \underline{single} element from $B$.
        
        The set $A$ is called the \emph{domain of $f$}, and the set $B$ is called the \emph{codomain of $f$}.
     \end{definition}
     
     The notation for a function is $f:A\rightarrow B$, and if $y=f(x)$ we say that $(x,y)\in f$.
     
\end{frame}

\begin{frame}{Functions}
    Let $f:A\rightarrow B$ be a function.
    \begin{itemize}
        \item Let $a\in A$, then $f(a)$ is called the \textit{\textbf{image}} of $a$ under $f$.
        \item Let $C\subseteq A$, then $f(C):=\{f(c)|c\in C\}$ is called the \textit{Image} of $C$, $Im(C)$.
        \item $f(A)\subseteq B$, the \textit{image} of $A$ is called the \textit{\textbf{range}} of $f$.
        \item Let $D\subseteq f(A)$, the set $\{x\in A | f(x) \in D\}$ is called the \textit{\textbf{preimage}} of $D$.
    \end{itemize}
\end{frame}

\begin{frame}{Functions}
    \begin{definition}
        Consider a function $f:\mathds{R}^n\rightarrow\mathds{R}$, the \textbf{Graph} of $f$, $Gr(f)$ is defined as:
        $$Gr(f):=\{(x,y)\in \mathds{R}^n\times\mathds{R}|y=f(x)\}$$
    \end{definition}
    
    \pause
    
    Note: More generally neither the domain needs to be $\mathds{R}^n$ nor the codomain needs to be $\mathds{R}$, the case given above is just the most common situation in economics.
\end{frame}

\begin{frame}{Functions}
    
    \begin{definition}
        Consider $f:\mathds{R}\rightarrow\mathds{R}$ and $g:\mathds{R}\rightarrow\mathds{R}$,
        \begin{enumerate}
            \item Sum: $(f+g):\mathds{R}\rightarrow\mathds{R}$, and $(f+g)(x)=f(x)+g(x)$.
            \item Product: $(f\cdot g):\mathds{R}\rightarrow\mathds{R}$, and $(f\cdot g)(x)=f(x)g(x)$
            \item Division: $(f/g):\mathds{R}\rightarrow\mathds{R}$, and $(f/g)(x)=\frac{f(x)}{g(x)}$. This is only well defined when $g(x)\neq 0$.
            \item Scaling: If $\alpha\in\mathds{R}$, $(\alpha f):\mathds{R}\rightarrow\mathds{R}$, and $(\alpha f)(x)=\alpha f(x)$
        \end{enumerate}
    \end{definition}
    
\end{frame}

\begin{frame}{Functions}
    \begin{definition}
        Consider the functions $f:B\rightarrow C$, and $g:A\rightarrow B$, then the \textbf{composite} function $f \circ g:A\rightarrow C$ is defined as
        $$(f\circ g)(x) = f(g(x))$$
    \end{definition}
\end{frame}

\begin{frame}{Functions}
    \begin{definition}
        Consider sets $A$ and $B$, and the function $f:A\rightarrow B$.
        \begin{enumerate}
            \item $f$ is \textbf{injective} if, for $a$ and $a'$ in $A$, such that $a\neq a'$, then $f(a)\neq f(a')$.
            \item $f$ is \textbf{surjective} if, for any $b\in B$, exists $a\in A$ such that $f(a)=b$.
            \item $f$ is \textbf{bijective} if it is both, injective and surjective at the same time.
        \end{enumerate}
    \end{definition}
\end{frame}

\begin{frame}{Quick Quiz - 5 Minutes}
    Classify the following functions
    \begin{center}
        \begin{tabular}{c|c}
            Function & Classification \\
            \hline
            $f:\mathds{R}\rightarrow\mathds{R}, f(x) = x^2$ & \onslide<2->{-} \\
            $f:\mathds{R}\rightarrow[-1,1], f(x) = sin(x)$ & \onslide<3->{Surjective} \\
            $f:\mathds{R}\rightarrow\mathds{R}, f(x) = x^3$ & \onslide<4->{Bijective}
        \end{tabular}
    \end{center}
\end{frame}

\begin{frame}{Functions}
    \begin{proposition}
        If $f:A\rightarrow B$ is a bijective function, then there exists a unique function $g:B\rightarrow A$, bijective, such that $$g(f(x))=x$$
        
        $g$ is called the inverse of $f$, also known as $f^{-1}$.
    \end{proposition}
    
    \begin{proposition}
        Let $f:B\rightarrow C$, and $g:A\rightarrow B$ be both invertible functions, then $f\circ g$ is invertible. Moreover,
        
        $$(f\circ g)^{-1}=g^{-1}\circ f^{-1}$$
    \end{proposition}
    
\end{frame}

\begin{frame}{Functions}
    
    \begin{proof}
    
    Existence:
    
        \begin{itemize}
        
        \item Let $g = \{(b,a)|(a,b)\in f\}$. \pause
        
        \item If $(b,a_1),(b,a_2)\in g$, then $(a_1,b),(a_2,b)\in f$, but $f$ is injective, so $a_1 = a_2$. Then $g$ is a function.\pause
        
        \item The domain of $g$ is $\{b|(b,a)\in g\} = \{b|(a,b)\in f\} = f(X)$.\pause
        
        \item Let $(b,a_2)\in g$ and $(a_1,b)\in f$. Then $(a_2,b)\in f$, and given $f$ injective, we have $a_1=a_2$. Then $g\circ f = \{(a,a)|a\in A\} = Id.$\pause
        
        \item Let $f^{-1} = g$
        
        \end{itemize}
        
        Homework, show that $g$ is bijective. Hint: Go with contradiction.
    \end{proof}
    
\end{frame}

\begin{frame}{Functions}
    Unicity,
    
    \begin{itemize}
        \item Let $g$ and $h$ be inverse of $f$.\pause
        
        \item Assume $g(b)\neq h(b)$, at least for some $b\in B$.\pause
        
        \item As $b\in B$, then there is $a$ such that $f(a)=b$.\pause
        
        \item So $g(b) \neq h(b)$, but $g(f(a)) \neq h(f(a))$.\pause
        
        \item But $g\circ f$ and $h\circ f$ are both the identity so...\pause
        
        \item $g(f(a)) = a \neq a = h(f(a))$, contradiction!
        
    \end{itemize}
    
    So the inverse must be unique.
    
\end{frame}

\begin{frame}
    
    Composite Invertible
    \begin{proof}
    
    \begin{itemize}
        
        \item $(g\circ f)\circ(f^{-1}\circ g^{-1}) = g \circ f \circ f^{-1} \circ g^{-1}$. \pause
        
        \item $g \circ f \circ f^{-1} \circ g^{-1} = g \circ Id \circ g^{-1} $. \pause
        
        \item $g \circ Id \circ g^{-1} = g \circ g^{-1} = Id. $ \pause
    \end{itemize}

    Trivial to show that $(f^{-1}\circ g^{-1})\circ (g\circ f) = Id.$ as well, using the same steps.
    
    \end{proof}
\end{frame}

\begin{frame}{Functions}
\begin{definition}
    Consider $f:\mathds{R}^n\rightarrow \mathds{R}$ and $\hat{y}$ in the codomain. 
    \begin{enumerate}
    \item The \textbf{level curve} of $f$ at $\hat{y}$ is:
    
    $$\mathcal{C}_{\hat{y}}=\{(x,\hat{y})\in \mathds{R}^{n+1}|f(x)=\hat{y}\}$$
    \item The \textbf{isoquant} curve of $f$ at $\hat{y}$ is:
    $$I_{\hat{y}}=\{x\in\mathds{R}^n|f(x)=\hat{y}\}$$
    \end{enumerate}    
\end{definition}
\end{frame}

\begin{frame}{Functions}
    \begin{definition}
        Consider the function $f:\mathds{R}\rightarrow\mathds{R}$, and any pair $x$ and $y$ in $\mathds{R}$ such that $x<y$, we say that
        \begin{enumerate}
            \item $f$ is \textbf{increasing} if $f(x)\leq f(y)$.
            \item $f$ is \textbf{decreasing} if $f(x)\geq f(y)$.
        \end{enumerate}
    \end{definition}
    
    If the inequalities are strict, then you add the word \emph{strictly} to increasing or decreasing. A non decreasing function is also known as monotonically increasing. Conversely, a non increasing function is also known as monotonically decreasing.
\end{frame}

\begin{frame}{Functions}
    \begin{definition}
        Consider the function $f:\mathds{R}^n\rightarrow\mathds{R}$, and any pair $x$ and $y$ in $\mathds{R}^n$ such that $y_i=x_i$ for every $i=1,.,j-1,j+1,..,n$, and $y_i=x_i+\epsilon$, with $\epsilon>0$ we say that
        \begin{enumerate}
            \item $f$ is \textbf{increasing in the component $j$} if $$f(x_1,\ldots,x_j,\ldots,x_n)\leq f(x_1,\ldots,x_j+\epsilon,\ldots,x_n)$$
        \end{enumerate}
    \end{definition}
    
    If the inequalities are strict, then you add the word \emph{strictly} to increasing or decreasing.
\end{frame}


\end{document}
