\documentclass[answers]{exam}
\usepackage[utf8]{inputenc}
\usepackage{amsmath}
\usepackage{amsfonts}

\title{Problem Set 1\\ Sets and Functions}
\author{Paulo Fagandini}
\date{}

\begin{document}

\maketitle

\begin{questions}

\section*{Sets}

\question Consider sets $A$, $B$, and $C$. Using the definition of ``subset'' show that:

\begin{parts}
\part $A\subseteq A$.

\begin{solution}
    Note that $\forall x\in A$, we have that $x\in A$, and therefore $A\subseteq A$.
\end{solution}

\part If $A\subseteq B$, and $B\subseteq C$, then $A\subseteq C$.

\begin{solution}
If $A\subseteq B$, then $\forall x \in A$ we have that $x\in B$. Now, if $B\subseteq C$, then $\forall z  \in B$, we have that $z\in C$, and as $x\in B$, then $x\in C$. Therefore $\forall x \in A$, $x\in C$, which means that $A\subseteq C$.

\end{solution}

\part If $A\subseteq B$, and $B\subseteq A$, then $A=B$.

\begin{solution}
First part, if $A\subseteq B$, then $\forall x\in A\Rightarrow x\in B$. Now as $B\subseteq A$ as well, then $\forall x \in B\Rightarrow x\in A$. Then $\forall x $, $x\in A \Leftrightarrow x\in B$, or $A=B$.
\end{solution}

\part $\emptyset\subseteq A$, and $A\subseteq\mathcal{U}$.

\begin{solution}
    $\emptyset\subseteq A$ if $\forall z\in \emptyset \Rightarrow z\in A$. But there is no $z$ in $\emptyset$, so $\forall z, z\in \emptyset$ is false, and by logic, false implies anything, in particular $z\in A$, and therefore $\emptyset\subseteq A$.
    Now, $A\subseteq \mathcal{U}$ implies that $\forall x\in A\Rightarrow x\in\mathcal{U}$. By definition $\mathcal{U}$ contains all the elements, in particular $x$, so $A\subseteq\mathcal{U}$.
\end{solution}

\end{parts}

\question Consider the sets $A$, $B$, and $C$. Show that:

\begin{parts}
\part $A\times \emptyset = \emptyset \times A = \emptyset$
\part $A\times B=B\times A \Leftrightarrow A=B$
\part $A\times(B\cup C)=(A\times B)\cup(A\times C)$
\part $A\times(B\cap C)=(A\times B)\cap(A\times C)$
\end{parts}

\question Consider sets $A$, $B$, and $C$. Show that:

\begin{parts}
\part $A\cap B \subseteq A$
\part $(A\cap B)^c = A^c \cup B^c$
\part Is it true that if $A\cap B=\emptyset$, and $B\cap C=\emptyset$, then $A\cap C=\emptyset$?
\part Find set $A$ such that $A\subseteq A\times A$.
\end{parts}

\question Given sets $A$ and $B$, such that $\#A=m$ and $\#B=n$,
\begin{parts}
\part Find $\#(A\times B)$.
\part Find $\#(A^k)$.
\part Find $\#\mathcal{P}(A\times B)$.
\part Show that $\#(A\cup B) = \#A+\#B -\#(A\cap B)$.
\end{parts}

\section*{Real Numbers}

\question Let $x,y\in\mathbb{R}$. Show that:

\begin{parts}
\part $|x|\geq 0$
\part $|x+y|\leq |x|+|y|$
\part $|x|-|y|\leq|x-y|$
\part $|xy|=|x||y|$
\end{parts}

\question Solve the following inequalities,

\begin{parts}
\part $|x-3|<9$

\begin{solution}
    $x\in (-6,12)$
\end{solution}

\part $|x-1|+|x-2|\geq 1$
\begin{solution}
    Given that $|\cdot|\geq0$, we have that the left hand side $x\in(-\infty,0]\cup[2,\infty)$, and the right hand side $x\in(-\infty,1]\cup[3,\infty)$ So obviously for $x\in(-\infty,1]\cup[2,\infty)$ satisfies the inequality. Note that for $x\in(1,2)$ both inequalitie
\end{solution}
\part $|x-1|+|x+1|<2$
\end{parts}

\question Let $x,y\in\mathbb{R}$. Prove that:
\begin{parts}
\part $max(x,y)=\frac{x+y+|y-x|}{2}$
\part $min(x,y)=\frac{x+y-|y-x|}{2}$
\end{parts}


\question Let $S=[0,1]$, interval, and $A_s=\left[0,\frac{1}{1+s}\right]$.

\begin{parts}
\part Find $\bigcap_{s\in S} A_s$.
\part Find $\bigcup_{s\in S} A_s$
\end{parts}

\question Consider $x_0\in\mathbb{R}$ and $\delta\in\mathbb{R}_{++}$. Let $A_\delta =\{x\in\mathbb{R}| \quad |x-x_0|\leq\delta\}$. Show that $\bigcap_{\delta>0} A_{\delta}=\{x_0\}$

\question Show that $A\subseteq \mathbb{R}$ is bounded if and only if there is $c\in\mathbb{R}$ such that for any $x\in A$, $|a|\leq c$.

\question Find $max(),min(),sup(),inf()$ if applicable for the following sets:
\begin{parts}
\part $A=\{x\in\mathbb{R}|\quad |x+a|-2x\leq 4\}$
\part $A = \{x\in\mathbb{R}| \quad x^2-3x+2\geq 0\}$
\part $A=\{x\in\mathbb{R}|\quad |x-2a|-|x+a|>12\}$
\end{parts}

\section*{Functions}

\question Consider $f,g:\mathbb{R}\rightarrow\mathbb{R}$, with $f(x)=x^2-4x+3$ and $g(x)=e^{2x^2}$, find:

\begin{parts}
\part $f+2g$
\part $f\circ g$
\part $g\circ g$
\part $\frac{f}{f\circ f}$
\part $f\cdot g$
\end{parts}

\question Show that, in general, $f\circ g\neq g\circ f$.

\question Let $f(x)=ax+b$, such that $f\circ f(x)=4x+3$. Find $f(5)$.

\question Let $f(x)=7x+2$, find $g(x)$ such that $f\circ g (x)=x$.

\question Show that $f:\mathbb{R}\rightarrow\mathbb{R}$, with $f(x)=e^x$ is injective but not bijective.

\question Given $\alpha>0$, find if the function $f(x)=x^\alpha$ is injective, surjective or bijective.

\question Show that $f(x)=ax^2+bx+c$ is not injective in $\mathbb{R}$.

\question Given $\alpha>0$, show that $f(x)=x^\alpha$ is invertible in $\mathbb{R}_{++}$.

\question Find the isoquants at $y_0>0$ for the following functions:

\begin{parts}
\part $f(x_1,x_2)=x_1^2+x_2^2$.
\part $f(x_1,x_2)=x_1^2\cdot x_2^2$.
\part $f(x_1,x_2)=\max\{x_1,x_2\}$.
\end{parts}

\question Show that if $f$ is strictly increasing or decreasing, it must be injective.

\question Provide a function that while being injective, is not strictly increasing or decreasing.

\question Show that if $f$ and $g$ are strictly increasing functions, then $f\circ g$ is also strictly increasing.

\question If $f:\mathbb{R}\rightarrow\mathbb{R}$ is strictly increasing and invertible, what can you say about the increasingness of $f^{-1}$?

\question Show that if the real valued function $f(x)$ is convex, then $g=-f(x)$ is concave.

\question Show that if $g>0$ and $f$ is increasing then $$\frac{f(x+h)-f(x)}{h}>0$$

\question Given $f:\mathbb{R}\rightarrow\mathbb{R}$, and $A\subseteq \mathbb{R}$, define $f(A):=\{f(a)| a\in A\}$. Let $S=sup(A)$. If $f$ is strictly increasing, is it true that $f(S)=sup(f(A))$?
\end{questions}


\end{document}
