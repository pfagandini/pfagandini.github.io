\documentclass[a4paper,12pt,answers]{exam}

\usepackage[margin=1in]{geometry}
\usepackage{amsmath}
\usepackage{amsfonts}
\usepackage{dsfont}
\usepackage{graphicx}
\usepackage{xcolor}
\usepackage{tikz}

\usepackage[defaultsans,oldstyle,scale=0.95]{opensans}
\usepackage[T1]{fontenc}

\usepackage[black]{PlayfairDisplay} 
\renewcommand*\oldstylenums[1]{{\playfairOsF #1}}

\renewcommand{\familydefault}{\sfdefault}

\title{Pre-Exam}
\author{Paulo Fagandini\\ Nova SBE}
\date{}

\headrule
\header{\includegraphics[scale=0.25]{NovaSecundarioV4.png}}{Bridging Mathematics 4509}{PhD Econ | Fin}

\footrule
\footer{}{Page \thepage}{}

\begin{document}

\maketitle
\thispagestyle{headandfoot}

\makebox[\textwidth]{Name:\enspace\hrulefill}
\vspace{0.1cm}

\begin{itemize}
	\item Remember, there are no ETCs associated to this course, so the \textit{grade} will not affect your GPA.
	\item This assessment is closed book and individual.
	\item You have 90 minutes to solve it.
\end{itemize}

\section*{Questions}

{
\sffamily
\begin{questions}

\question Solve, for some $c\in\mathds{R}$ $$\frac{\partial k}{\partial x}=k(x)+c$$

    \begin{solutionorgrid}[11cm]
        $$\frac{dk/dx}{k(x)+c}=1$$
        $$\frac{dk}{k(x)+c}=dx$$
        $$\int \frac{1}{k(x)+c} dk=\int 1 dx$$
        $$ln(|k(x)+c|)+A_0 =  x + A_1$$
        with $A_0$ and $A_1$ some constants. Let $A_2=A_1-A_0$:
        $$ln(|k(x)+c|) =  x + A_2$$
        $$|k(x)+c|=e^{x+A_2}=e^xe^{A_2}$$
        Let $e^{A_2}=A$
        $$k(x)+c = \pm A e^x$$
        $$k(x)=\pm Ae^x - c$$
        Given that $A$ is an arbitrary condition, which will be determined by $x_0$, then we can get rid of the $\pm$, so:
        $$k(x)=Ae^x-c$$
    \end{solutionorgrid}

\question From $x(t)+e^{x(t)}=t$ find $dx/dt$

    \begin{solutionorgrid}[13cm]
        Differentiate both sides on $t$
        $$\frac{\partial}{\partial t} \left[x(t)+e^{x(t)}\right] =  \frac{\partial}{\partial t}t $$
        $$\frac{\partial x(t)}{\partial t} + e^{x(t)}\frac{\partial x(t)}{\partial t} = 1$$
        $$\frac{\partial x(t)}{\partial t}\left[1+e^{x(t)}\right] = 1$$
        $$\frac{\partial x(t)}{\partial t}=\frac{1}{1+e^{x(t)}}$$
        Which concludes the exercise.
    \end{solutionorgrid}

\question Consider a sample of $n$ numerical random variables, who are independently and indetically distributed, following a continuous distribution with pdf $f(x)$ and cdf $F(x)$. Write down the expected value of the highest order statistic.\footnote{If you do not know what this is, it is the expected value of the highest value in the sample.}

    \begin{solutionorgrid}[8cm]
        $$E[x_n] = \frac{\int_{\mathcal{S}} x f(x)F(x)^{n-1} dx}{\int_{\mathcal{S}} f(x)F(x)^{n-1} dx}$$
        The relevant part here, or maybe novel, is the denominator. Note here that if we do not do it, then our ``density'' does not add up to 1.
    \end{solutionorgrid}

\question Explain the difference between maximum, supremum, and upper bound of some set $\mathcal{S}$

    \begin{solutionorgrid}[6cm]
        \begin{itemize}
            \item An upper bound is any number that is larger or equal than any $s\in\mathcal{S}$.
            \item The supremum is the lowest upper bound.
            \item The maximum is the supremum when this is included in $\mathcal{S}$
        \end{itemize}        
    \end{solutionorgrid}


\question Let $(X,d)$ be a metric space, and let $B(x_0,r)$ denote the open ball centered at $x_0$ with radius $r>0$. Prove that the open ball $B(x_0,r)$ is an open set in the metric space $(X,d)$.

    \begin{solutionorgrid}[15cm]
        The definition of an open set is that for any of its elements you can create an open ball that is fully contained in the set.

        Given that an open ball is $$B(x_0,r)=\{x\ |\ d(x,x_0)<r\}$$

        Take $a\in B(x_0,r)$, meaning that $d(a,x_0)<r$. You can easily construct $B(a,r-d(a,x_0))$. Note that this ball is fully contained in the original ball.

        Take $y\in B(a,(r-d(a,x_0)))$. Because of the triangular inequality we have:
        $$d(x_0,y)\leq d(x_0,a)+d(a,y)<d(x_0,a)+r-d(a,x_0)=r$$ therefore we have that $$d(x_0,y)<r$$ or in other words, $y\in B(x_0,r)$ completing the proof.
    \end{solutionorgrid}

\question Consider the space of Cobb Douglas functions, \textit{i.e.}, the functions of the form $f(x)=Ax^\alpha y^\beta$. Is this a vector space?

    \begin{solutionorgrid}[20cm]
        No. For it to be a vector space, it is necessary that the addition of two elements of this space is an element of the space, however $A_1 x^{\alpha_1}y^{\beta_1}+A_2 x^{\alpha_2}y^{\beta_2}$ cannot be written as a Cobb-Douglas function, unless for very specific values of $(\alpha,\beta)$.
    \end{solutionorgrid}

\end{questions}
}

\clearpage

\center{DRAFT}

\clearpage

\center{DRAFT}

\end{document}