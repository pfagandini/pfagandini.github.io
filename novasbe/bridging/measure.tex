\documentclass[aspectratio=169,handout]{beamer}
\usepackage[utf8]{inputenc}
\usepackage{amsmath}
\usepackage{tikz}
\usepackage{dsfont}
\usepackage{opensans}
\usepackage{dsfont}

\usetheme{NOVASBE}

\title[]{4509 - Bridging Mathematics}
\subtitle{Introduction to Measure Theory\\ and Integration}
\author[P. Fagandini]{Paulo Fagandini}
\institute{}
\date{}

\newtheorem{defenition}{Definition}[section]
\newtheorem{proposition}{Proposition}[section]
\newtheorem{conjecture}{Conjecture}[section]

\begin{document}

\section{$\sigma$-Algebras}

\begin{frame}
   \begin{definition}
        Let $X$ be a set. An \textit{algebra} is a collection $\mathcal{A}$ of subsets of $X$ such that:
        \begin{enumerate}
            \item $\emptyset \in \mathcal{A}$
            \item If $A$ \in $\mathcal{A}$, then $A^c\in\mathcal{A}$
            \item If $A_1,A_2,...,A_n\in\mathcal{A}$ then $\bigcup_{i=1}^n A_i\in\mathcal{A}$ and $\bigcap_{i=1}^n A_i\in\mathcal{A}$
        \end{enumerate}
        If 3 holds for countable infinite sets $A_i$ (\textit{i.e.} you replace the $n$ by $\infty$), then $\mathcal{A}$ is a $\sigma-algebra$.
   \end{definition}
\end{frame}

\begin{frame}{Example}

    These are examples for $\mathcal{A}$ being a $\sigma-algebra$

    \begin{enumerate}
        \item<1-> Let $X=\mathds{R}$, and $\mathcal{A}$ the set of all the subsets of $\mathds{R}$.
        \item<2-> Let $X=[0,1]$ and let $\mathcal{A}=\left\{\emptyset, X, \left[0,\frac{1}{2}\right], \left(\frac{1}{2},1\right]\right\}$
    \end{enumerate}
\end{frame}

\begin{frame}
    \begin{definition}
        The pair $(X,\mathcal{A})$ is called a \textit{measurable space}. A set $A$ is \textit{measurable} if $A\in\mathcal{A}$    
    \end{definition}
\end{frame}

\begin{frame}
    \begin{lemma}
        If $\mathcal{A}_\alpha$ is a $\sigma-algebra$ for each $\alpha\in I$, with $I$ an index set, then $\cap_{\alpha\in I}\mathcal{A}_\alpha$ is a $\sigma-algebra$
    \end{lemma}
    
    \pause

    \begin{proof}
        \begin{enumerate}
            \item If $\mathcal{A}_\alpha$ is a $\sigma-algebra$, therefore $\emptyset\in\mathcal{A}_\alpha$ $\forall \alpha\in I$, and then $\emptyset \in \cap_{\alpha\in I}\mathcal{A}_\alpha$
            \item Let $S_i\in \cap_{\alpha\in I}\mathcal{A}_\alpha$. It follows that $S_i\in\mathcal{A}_\alpha$ $\forall \alpha\in I$, and also $S_i^c\in \mathcal{A}_\alpha$ $\forall \alpha\in I$, but then $S_i^c\in \cap_{\alpha\in I}\mathcal{A}_\alpha$.
            \item Choose a collection $\{S_i\}_i^\infty\in \cap_{\alpha\in I}\mathcal{A}_\alpha$. Now given that $S_i$ must also be in every $\mathcal{A}_\alpha$, their intersection is also in $\mathcal{A}_\alpha$, and therefore it must be in $\cap_{\alpha\in I}\mathcal{A}_\alpha$.
        \end{enumerate}
    \end{proof}
\end{frame}

\begin{frame}
    Let $\mathcal{C}$ be a collection of subsets of $X$, define: 
    \begin{align*}
        \sigma(\mathcal{C})=\cap\left\{\mathcal{A}_\alpha | \mathcal{A}_\alpha\ \text{is a }\sigma-algebra,\ \mathcal{C}\subset\mathcal{A}_\alpha\right\}
    \end{align*}
    this is, the intersection of all $\sigma-algebras$ containing $\mathcal{C}$. Note that $\sigma(\mathcal{C})$ is non empty, as at least the $\sigma-algebra$ $\mathcal{P}(X)$ contains $\mathcal{C}$. Using the previous lemma, we have that $\sigma(\mathcal{C})$ is itself a $\sigma-algebra$. We call this the \textit{$\sigma-algebra$ generated by $\mathcal{C}$}, or that $\mathcal{C}$ generates the $\sigma-algebra$ $\sigma(\mathcal{C})$.
\end{frame}

\begin{frame}
    \begin{fact}
        Continuing with the previous definition we can state that:
        \begin{enumerate}
            \item If $\mathcal{C}_1\subset \mathcal{C}_2$, then $\sigma(\mathcal{C}_1)\subset\sigma(\mathcal{C}_2)$.
            \item Since $\sigma(\mathcal{C})$ is a $\sigma-algebra$, then $\sigma\left(\sigma(\mathcal{C})\right)=\sigma(\mathcal{C})$.
        \end{enumerate}
    \end{fact}
\end{frame}

\begin{frame}
    \begin{definition}
        If $X$ has some structure, for example if it is a metric space, then we can consider open sets in $X$. If $\mathcal{G}$ is the collection of open subsets of $X$, then $\sigma(\mathcal{G})$ is the \textbf{Borel $\sigma-algebra$} on $X$, and it is denoted as $\mathcal{B}$. The elements of $\mathcal{B}$ are called \textit{Borel sets}, and are said to be \textit{Borel measurable}.    
    \end{definition}
\end{frame}

\begin{frame}
    \begin{proposition}
        If $X=\mathbb{R}$, then the Borel $\sigma-algebra$ $\mathcal{B}$ is generated by each of the following collection of sets:
        \begin{enumerate}
            \item $\mathcal{C}_1=\{(a,b)|a,b\in\mathbb{R}\}$
            \item $\mathcal{C}_2=\{[a,b]|a,b\in\mathbb{R}\}$
            \item $\mathcal{C}_3=\{(a,b]|a,b\in\mathbb{R}\}$
            \item $\mathcal{C}_4=\{(a,\infty)|a,b\in\mathbb{R}\}$
        \end{enumerate}
    \end{proposition}
\end{frame}

\begin{frame}
    \begin{proof}
        \begin{enumerate}
            \item Let $\mathcal{G}$ be the collection of open sets. By definition $\sigma(\mathcal{G})$ is the Borel $\sigma-algebra$. Since every element of $\mathcal{C}_1$ is open, then $\mathcal{C}_1\subset\mathcal{G}$, and consequently $\sigma(\mathcal{C}_1)\subset \sigma(\mathcal{G})=\mathcal{B}$.
        \end{enumerate}
        
    \end{proof}
    
\end{frame}

\section{Measures}

\section{Lebesgue Integral}

\end{document}
